\documentclass[a4paper]{article}
\usepackage[utf8]{inputenc}
\usepackage{indentfirst}
\usepackage{polski}
\usepackage[left=2.5cm,right=2.5cm,top=2cm,bottom=2cm]{geometry}
\usepackage{graphicx}
\usepackage{listings}
\linespread{1.3}
\usepackage{float}
\usepackage{enumitem}

\title{\vspace{80mm}Implementacja interfejsu 1wire w VHDL przy użyciu Spartan-3E oraz DS18S20}
\author{Krzysztof Cabała 210047\\ Kinga Wilczek 210063\vspace{110mm}}

\begin{document}
\maketitle
\thispagestyle{empty}

\tableofcontents
\newpage
\addcontentsline{toc}{section}{Spis rysunków}
\listoffigures

\newpage

\section{Założenia projektowe}

Celem projektu było przygotowanie układu obsługującego sensor Dallas DS18S20 na platformie Xilinx Spartan-3E. W tym celu zaimplementowano obsługę magistrali One wire oraz interpretacje wyniku. Efektem działania układu jest wyświetlona wartość temperatury na wyświetlaczu LCD zgodnym ze standardem HD44780.

\section{Wstęp teoretyczny}
\subsection{Interfejs}
Interfejs OneWire został opracowany przez firmę Dallas Semiconductor \cite{bib_kom} do komunikacji między dwoma lub większą liczbą urządzeń przy wykorzystaniu zaledwie jednej lini danych, lini GND (konieczne odniesienie dla poprawnego rozpoznawania stanów logicznych) oraz zasialania Vcc. 
W ramach oszczędności przewodów ogranicza się połączenia do dwóch linii, wtedy układ zasilany jest pasożytniczo z lini danych.
Przykładowe połączenie przedstawia schemat:

\begin{figure}[h]
\begin{center}
\includegraphics[scale=0.3]{graphics/idea.png}
\end{center}
\caption{Przykładowe połączenie urządzeń}
\label{schemonewire}
\end{figure}

Wyróżnia się urządzania typu Master (najczęściej mikrokontroler) oraz Slave (peryferia).

\subsection{Komunikacja}

\subsubsection{Inicjalizacja i reset}

Każda próba komunikacji urządzeń master i slave musi zacząć się od sekwencji składającej się z sygnału reset, wysyłanego przez master, po którym następuje sygnał obecności układu slave. Sygnał reset to wymuszony stan 0 trwający przynajmniej 480$\mu$s. Następnie master oczekuje na sygnał obecności innego urządzenia na linii. Następuje wówczas zwolnienie magistrali, co powoduje podciągnięcie jej do stanu wysokiego przez rezystor pull-up. Urządzenie slave wykrywa wówczas narastające zbocze linii i po upływie 15$\mu$s-60$\mu$s sygnalizuje swoją obecność poprzez wymuszenie stanu niskiego na okres 60$\mu$s-240$\mu$s.

\begin{figure}[H]
\begin{center}
\includegraphics[scale=0.4]{graphics/init.png}
\end{center}
\caption{Diagram czasowy dla procedury inicjalizacji}
\label{inititming}
\end{figure}

\subsubsection{Zapis bitu}
Operacje zapisu bitu realizowane są w ściśle określonych slotach czasowych. Długość jednego slotu wynosi zwykle 60$\mu$s. Próbkowanie dokonywane jest mniej więcej w środku slotu celem uodpornienia na błędy. Pomiędzy kolejnymi operacjami wymagana jest przynajmniej 1$\mu$s odstępu.

Zapis rozpoczyna się wysterowaniem linii danych przez master na poziom niski. Zapis 0 wymaga utrzymania jej w tym stanie przez cały slot. Zapis 1 jest nieco bardziej skomplikowany. Master musi w czasie nie dłuższym niż 15$\mu$s, ale nie krótszym niż 1$\mu$s zwolnić magistralę tak aby w momencie próbkowania (po 30$\mu$s od zbocza opadającego) była w stanie wysokim. 

\subsubsection{Odczyt bitu}
Odczyt bitu również wymaga 60$\mu$s slotu oraz 1$\mu$s przerwy. Odczyt rozpoczyna się wymuszeniem przez master stanu niskiego na linii danych na czas nie krótszy niż 1$\mu$s i zwolnienie jej (powrót do stanu wysokiego). Po tym sygnale sterowanie linią przejmuje urządzenie slave, wysyłające bit 0 lub 1. Slave po wykryciu zbocza opadającego wymusza stan niski (dla 0) lub utrzymuje  wysoki (dla 1) linii danych. Sygnał musi być wtedy spróbkowany przez master. Przed upłynięciem czasu końca slotu maigistrala zostaje zwolniona przez slave, co powoduje jej powrót do stanu wysokiego.

\begin{figure}[H]
\begin{center}
\includegraphics[width=14cm]{graphics/slots.png}
\end{center}
\caption{Diagramy czasowe dla procedury zapisu i odczytu}
\label{slotsitming}
\end{figure}

\subsection{Konwersja i odczyt temperatury}
Aktualny odczyt temperatury zapisany jest w dwóch pierwszych bajtach pamięci Scratchpad \cite{bib_ds18s20}. Po poprawnej inicjalizacji czujnika DS18S20 ich zawartość odpowiada temperaturze +85$^\circ$C. W celu zmierzenia aktualnej temperatury należy wysłać do termometru komendę konwersji, zresetować układ i odczytać pamięć. W pamięci urządzenia jest 9 bajtów wyniku. Tylko dwa pierwsze są istotne z punktu widzenia wyniku pomiaru. Pierwszy uzyskany bajt określa znak odczytu. Natomiast ostatni bit drugiego bajtu stanowi o wystąpieniu cyfry po przecinku. Pozostałe bity wyniku to wartość odczytanej temperatury.  Wartość ta następnie podlega konwersji na kod BCD za pomocą  algorytmu double dabble.

\subsection{Double dabble}
Double dabble jest powszechnie wykorzystywanym algorytmem do konwersji liczb binarnych na kod BCD (Binary-Coded Decimal - system dziesiętny zakodowany dwójkowo) \cite{bib_dabble}. 

Algorytm polega na wykonaniu n iteracji (w zależności od długości ciągu bitów). Początkowo wynikowy kod BCD jest zainicjalizowany jako ciąg zer podzielonych na grupy po 4 bity. Podczas każdej iteracji wykonywane jest przesunięcie o jeden bit w lewo, a na koniec doklejany jest jeden bit ciągu wejściowego. Jeżeli przed kolejnym przesunięciem wartość w jednej z grup w kodzie BCD jest wyższa niż 4 to następuje dodanie binarnej 3. Po wykonaniu odpowiedniej ilości iteracji algorytm kończy swoje działanie. Ostatecznie wynikowy ciąg jest podzielony po cztery bity, które odpowiadają kolejnym cyfrom wyniku. 

\section{Implementacja}

\subsection{Wykorzystanie układu Spartan 3e}

Implementacji układu dokonano na platformie Spartan 3e \cite{bib_spartan}. Raport wygenerowany przez oprogramownie Xilinx ISE (rysunek \ref{raport}) pokazuje zużycie zasobów na poziomie 6\% co daje możliwości dalszej rozbudowy projektu.

\begin{figure}[H]
\begin{center}
\includegraphics[width=14cm]{graphics/summary.png}
\end{center}
\caption{Wykorzystanie zasobów układu Spartan 3e}
\label{raport}
\end{figure}

Raport czasowy określa maksymalną częstotliwość pracy układu na poziomie 57.071MHz, co gwarantuje jego poprawne zachowanie przy wbudowanym taktowaniu 50MHz.

\subsection{Schemat główny}

Funkcjonalność projektu opiera się na 6 podstawowych modułach widocznych na rysunku \ref{top_level}:
\begin{itemize}
\item Termometer - obsługa sensora DS18S20;
\item RotaryEnc- sterowanie częstotliwością pomiaru;
\item Interpreter - interpretacja 2 bajtowego wyniku;
\item double\_dabble - zamiana liczby binarnej bez znaku na kod BCD;
\item LCD - generowanie zawartości wyświetlacza LCD;
\item LCDWrite - sterownik wyświetlacza.
\end{itemize}

\begin{figure}[H]
\begin{center}
\includegraphics{graphics/top_level.png}
\end{center}
\caption{Schemat ogólny projektu}
\label{top_level}
\end{figure}

\subsection{Termometer}
Moduł Termometer jest najbardziej rozbudowanym modułem składającym się z 4 układów (rysunek \ref{termometer}):
\begin{itemize}
\item Controller - obsługa sekwencji komunikacji i konwersji temperatury;
\item ByteModule - obsługa komunikacji na poziomie bajtu (instrukcji);
\item BusController - obsługa magistrali OneWire;
\item IOBuf - ustala kierunek komunikacji.
\end{itemize}

\begin{figure}[H]
\begin{center}
\includegraphics[height=18cm]{graphics/termometer.png}
\end{center}
\caption{Moduł Termometer}
\label{termometer}
\end{figure}

\subsubsection{Controller}


Moduł Controller (rysunek \ref{controller_sym}) odpowiada za przeprowadzenie poprawnej sekwencji operacji wymaganych do zmierzenia i odczytu temperatury.

\begin{figure}[H]
\begin{center}
\includegraphics[height=5cm]{graphics/controller_sym.png}
\end{center}
\caption{Moduł Controller}
\label{controller_sym}
\end{figure}

Opis wyprowadzeń:

Wejścia:
\begin{itemize}
\item CLK - Zegar;
\item Freq\_up - Impuls zwiększenia częstotliwości próbkowania;
\item Freq\_down - Impuls zmniejszenia częstotliwości próbkowania;
\item Byte\_in - Bajt odebrany przez ByteModule;
\item Bit\_in - Bit odebrany przez BusController;
\item Busy\_in - Flaga zajętości ByteModule;
\item c\_Busy\_in - Flaga zajętości BusController;
\end{itemize}

Wyjścia:
\begin{itemize}
\item Start - Sygnał start dla ByteModule;
\item Data\_out - Odebrany ciag reprezentujący temperaturę (2 bajty);
\item Byte\_out - Dane (kod instrukcji) dla ByteContorllera;
\item Reser - Sygnał reset;
\item Freq\_state - Wektor reprezentujący aktualną częstotliwość próbkowania;
\item Reset\_start - Sygnał start dla BusControllera. 
\end{itemize}

W celu ograniczenia częstotliwości odczytu (co powoduje nadmierne nagrzewanie się czujnika) dodano możliwość sterowania odstępem między pomiarami. Moduł umożliwia odczyt co 4, 2, 1 sekundę lub ciągły. Zmiana odbywa się w procesie freq\_process.
\newpage
\lstset{language=VHDL}
\begin{lstlisting}[frame=single]
freq_process: process (CLK, freq_up, freq_down)
begin
	if rising_edge(clk) then
		if freq_up = '1' then
			frequence <= frequence(2 downto 0) & frequence (3);
		elsif freq_down = '1' then
			frequence <= frequence (0) & frequence(3 downto 1);

		end if;
	end if;
end process;
\end{lstlisting}

Moduł jest automatem skończonym (fsm). Zmiana stanów odbywa się w procesie  state\_service.

\lstset{language=VHDL}
\begin{lstlisting}[frame=single]
state_service: process (CLK, next_state)
begin
	if (rising_edge (CLK)) then
		state <= next_state;
	end if;
end process;
\end{lstlisting}

Kolejne stany maszyny pokazuje Rysunek \ref{controller_fsm}. Po każdym ze stanów (poza presence) występuje dodatkowy stan oczekiwania na zwolnienie flagi busy (zajętości kontrolerów niższego rzędu). Stany te zostały pominięte na schemacie ze względu na jego czytelność. Należy zaznaczyć, że stany w których wykonwywane są kolejne komendy są jednotaktowe, podczas gdy oczekiwanie na ich wykonanie zajmuje stosunkowo dużo więcej czasu. Zauważyć to można na symulacji pokazanej na rysunku \ref{controller_symulation}

\begin{figure}[H]
\begin{center}
\includegraphics[height=10cm]{graphics/controller_fsm.png}
\end{center}
\caption{Maszyna stanów Controller}
\label{controller_fsm}
\end{figure}

\newpage

\begin{figure}[H]
\begin{center}
\includegraphics[height=22cm]{graphics/controller_symulation.png}
\end{center}
\caption{Symulacja automatu Controller}
\label{controller_symulation}
\end{figure}

Zaznaczona na symulacji pozycja A pokazuje moment zwolnienia magistrali przez master (stan wysokiej impedancji). Zaraz po jej zwolnieniu kontrolę przejmuje slave, który wymuszając stan niski sygnalizuje swoją obecność (moment B). Przybliżony fragment (czerwona ramka) obrazuje przejścia stanów pomiędzy reset\_slave\_b, a skip\_b. Widać, że stany presence oraz skip są jednotaktowe. 


\subsubsection{ByteModule}

ByteModule został zaimplementowany w celu obsługi dwukierunkowej komunikacji pomiędzy master, a slave. W zależności od kierunku transmitowanego bajtu wykonywane są na nim różne operacje. W przypadku zapisu danych bajt wejściowy jest rozdzielany na pojedyncze bity, które następnie kolejno są przesyłane do urządzenia. Natomiast operacja odczytu polega na formowaniu pojedynczych bitów wejściowych w jeden bajt wyjściowy. 

\begin{figure}[H]
\begin{center}
\includegraphics[height=5cm]{graphics/byte_module_sym.png}
\end{center}
\caption{Moduł ByteModule}
\label{bytemodule_sym}
\end{figure}

Opis wyprowadzeń:

Wejścia:
\begin{itemize}
\item Bit\_in - Bit wchodzący podczas operacji odczytu;
\item Start - Jednobitowy sygnał rozpoczęcia operacji;
\item RnW - Flaga informująca o typie wykonywanej operacji. Jeżeli ustawiona w stan wysoki wykonywany jest odczyt;
\item Busy\_bit - Flaga zajętości BusController;
\item Byte\_In(7:0) - Wejściowy bajt przy operacji zapisu;
\end{itemize}

Wyjścia:
\begin{itemize}
\item Byte\_Out(7:0) - Wyjściowy bajt przy operacji odczytu;
\item Start\_bit - Sygnał start dla modułu;
\item RnW\_bit - Bit określający rodzaj operacji wykonywanej przez BusController;
\item Busy - Bit zajętości ByteModule;
\item Bit\_out - Kolejny bit zapisywany przez BusController;
\end{itemize}

Ten moduł również jest maszyną stanów. Jej uproszczony schemat przedstawia rysunek \ref{byte_module_fsm}. Szczegóły implementacji można zobaczyć na symulacji przedstawionej na rysunkach \ref{byte_module_symulation1} - \ref{byte_module_symulation3}, 

\begin{figure}[H]
\begin{center}
\includegraphics[width=15cm]{graphics/byte_module_fsm.png}
\end{center}
\caption{Maszyna stanów BusController}
\label{byte_module_fsm}
\end{figure}

Cykl odczytu rozpoczyna stan rs w którym ustawiany jest bit RnW na wartość 1 oraz wysyłany jest jendotaktowy sygnał start\_bit. Następnie w stanie rb następuje oczekiwanie na sygnalizację zakończenia operacji przez BusController (flaga Busy\_bit). W stanie re w rejestrze tmp\_byte zatrzaskiwana jest odczytana wartość kolejnego bitu. Wartość ta jest współbieżnie przepisywana na wyjście byte\_read. W zależności od ilości odczytanych bitów następuje przejście do stanu rc zwiększania zmiennej counter lub stanu n oznaczającego oczekiwanie na następną operację. 

Za zapamiętywanie kolejnych bitów odpowiedzialny jest proces byte\_reading:

\lstset{language=VHDL}
\begin{lstlisting}[frame=single]
byte_reading: process (CLK, state)
begin
if rising_edge(CLK) then
	if state = re then
		tmp_byte <= Bit_in & tmp_byte(7 downto 1);
	end if;
end if;
end process;
\end{lstlisting}

Symulacja z rysunku \ref{byte_module_symulation1} pokazuje przebieg całej procedury. Zaznaczone czerowną i niebieską ramką fragmenty można zobaczyć w powiększeniu na rysunku \ref{byte_module_symulation2}. Widać na nich odpowiednio przejścia między odczytem kolejnych bitów oraz odczytem ostatniego bitu.

Procedura zapisu bajtu jest analogiczna do procedury odczytu. Wymienionym wcześniej stanom odpowiadają kolejno ws (RnW = 0), wb, we, wc. Wartości kolejnych bitów są odczytywane wprost z wektora byte\_in w zależności od wartości byte\_cnt.  Symulację przedstawia rysunek \ref{byte_module_symulation3}

\begin{figure}[H]
\begin{center}
\includegraphics[height=22cm]{graphics/byte_module_symulation1.png}
\end{center}
\caption{Symulacja zapisu bajtu}
\label{byte_module_symulation1}
\end{figure}

\begin{figure}[H]
\begin{center}
\includegraphics[height=22cm]{graphics/byte_module_symulation2.png}
\end{center}
\caption{Symulacja zapisu bajtu - szczegóły}
\label{byte_module_symulation2}
\end{figure}

\begin{figure}[H]
\begin{center}
\includegraphics[height=22cm]{graphics/byte_module_symulation3.png}
\end{center}
\caption{Symulacja odczytu bajtu}
\label{byte_module_symulation3}
\end{figure}

\newpage

\subsubsection{BusController}

Moduł BusController realizuje podstawowe operacje: reset, odczyt oraz zapis bitu.  

\begin{figure}[H]
\begin{center}
\includegraphics[height=5cm]{graphics/bus_controller_sym.png}
\end{center}
\caption{Moduł BusController}
\label{bus_controller_sym}
\end{figure}

Opis wyprowadzeń:

Wejścia:
\begin{itemize}
\item CLK - Wejście zegarowe;
\item Bus\_in - Bit odczytu z termometru;
\item Data\_id - Bit do zapisu z ByteModule;
\item Reset - Jednobitowy sygnał resetu;
\item RnW - Flaga informująca o typie wykonywanej operacji. Jeśli ustawiona w stan wysoki wykonywany jest odczyt;
\item Start - Sygnał start dla BusController.
\end{itemize}

Wyjścia: 
\begin{itemize}
\item Bus\_out - Zapis bitu do termometru; 
\item Busy - Bit zajętości BusController;
\item Read\_out -  Odczytany bit z termometru.
\end{itemize}

Poniżej przedstawiono schemat blokowy tego automatu (rysunek \ref{bus_controller_fsm}):

\begin{figure}[H]
\begin{center}
\includegraphics[width=15cm]{graphics/bus_controller_fsm.png}
\end{center}
\caption{Maszyna stanów BusController}
\label{bus_controller_fsm}
\end{figure}

Poprawna obsługa magistrali wymaga jej zwalniania poprzez ustawienie w stan wysokiej impedancji (podciągnięcie do Vcc przez rezystor pull-up). Służy do tego element IOBuf. Podanie logicznego zera na wejście T otwiera bufor wyjściowy i przekazuje sygnał podawany na pin I (GND - Logiczne 0). Logiczne 1 na wejściu T ustawia linie w stanie wysokiej impednacji (zwalnia magistralę). Pin O służy do odczytu stanu magistrali.

\begin{figure}[H]
\begin{center}
\includegraphics[scale=1.4]{graphics/io_buf_sym.png}
\end{center}
\caption{Symbol IOBuf}
\label{iobuf_controller_sym}
\end{figure}

\begin{figure}[H]
\begin{center}
\includegraphics[height=19cm]{graphics/bus_controller_symulation.png}
\end{center}
\caption{Symulacja odczytu bajtu}
\label{bus_controller_symulation}
\end{figure}

Na rysunku \ref{bus_controller_symulation} przedstawiono charakterystykę pracy układu wraz z buforem IO. Zaznaczono charakterystyne punkty:
\begin{enumerate}[label=\Alph*]
\item Zwolnienie magistrali przez master;
\item Sygnał obecności slave;
\item Zwolnienie magistrali przez slave;
\item Sygnał obecności slave na wyjściu modułu;
\item Impuls start;
\item Magistrala w stanie wysokim - odczyt 1;
\item Master wymusza stan niski - zapis 0;
\item Master zwalnia magistralę przy zapisie - zapis 1;
\end{enumerate}

\subsection{Interpreter}

\begin{figure}[H]
\begin{center}
\includegraphics[height=5cm]{graphics/interpreter_sym.png}
\end{center}
\caption{Moduł Interpreter}
\label{interpreter_sym}
\end{figure}

Interpreter jest modułem kombinacyjnym, którego zadaniem jest odpowiedni podział wyniku odczytu. Jako wejście przyjmowane są dwa bajty. W wyniku konwersji pierwszy bajt zmieniany jest na znak wartości temperatury. Kolejne 7 bitów jest wartością temperatury. Ostatni bit określa wartość po przecinku. 

Opis wyprowadzeń:

Wejścia:
\begin{itemize}
\item DATA(15:0) - dwa bajty wyniku uzyskanego z modułu termometer;
\end{itemize}

Wyjścia:
\begin{itemize}
\item SIGN - bit znaku, gdzie stan wysoki oznacza wystąpienie wartości ujemnej;
\item HALF - bit określający wartość po przecinku (0 lub 5). Jeżeli ustawiony na jeden wartość po przecinku wynosi 5;
\item VALUE(6:0) - wektor wartości temperatury po konwersji;
\end{itemize}

\subsection{Double dabble}

\begin{figure}[H]
\begin{center}
\includegraphics[height=5cm]{graphics/double_dabble_sym.png}
\end{center}
\caption{Moduł Double dabble}
\label{double_dabble_sym}
\end{figure}

Moduł ten realizuje konwersję uzyskanej wartości na kod BCD na podstawie wcześniej opisanego algorytmu double dabble. Użycie pętli for wymusiło wykorzystanie zmiennych zamiast sygnałów. Taka konstrukcja spowodowała wygenerowanie skomplikowanej logiki kombinacyjnej.

Opis wyprowadzeń:

Wejścia:
\begin{itemize}
\item BYTE\_IN(6:0) - wektor wejściowy konwertowany na BCD;
\end{itemize}

Wyjścia:
\begin{itemize}
\item S(3:0) - 4 bity cyfry setek;
\item D(3:0) - 4 bity cyfry dziesiątek;
\item J(3:0) -  4 bity cyfry jedności;
\end{itemize}

Algorytm double dabble realizowany jest w procesie bcd1:

\lstset{language=VHDL}
\begin{lstlisting}[frame=single]
bcd1: process(BYTE_IN)

	variable temp: STD_LOGIC_VECTOR (6 downto 0) ;
	variable jds: STD_LOGIC_VECTOR (11 downto 0):="000000000000";

begin	
	temp := BYTE_IN; 
	jds :=(others => '0');

		for i in 0 to 6 loop
			if jds(3 downto 0) > 4 then 
				jds(3 downto 0) := jds(3 downto 0) + 3;
			end if;
     
			if jds(7 downto 4) > 4 then 
				jds(7 downto 4) := jds(7 downto 4) + 3;
			end if;
				jds := jds(10 downto 0) & temp(6);
			 
			temp := temp(5 downto 0) & '0';
			
		end loop;
		
	J <= jds(3 downto 0);
	D <= jds(7 downto 4);
	S <= jds(11 downto 8);
end process bcd1; 
\end{lstlisting}
Przykład działania pokazje symulacja na rysunku \ref{double_dabble_symulation}:

\begin{figure}[H]
\begin{center}
\includegraphics[width=16cm]{graphics/double_dabble_symulation.png}
\end{center}
\caption{Symulacja modułu double dabble}
\label{double_dabble_symulation}
\end{figure}

\subsection{LCD}

\begin{figure}[H]
\begin{center}
\includegraphics[width=4cm]{graphics/lcd_sym.png}
\end{center}
\caption{Moduł LCD}
\label{lcd_sym}
\end{figure} 

Moduł LCD opiera się na prostej maszynie stanów, która przesyła kolejne znaki ASCII do wyświetlacza kontrolowanego przez element LCDWrite. Ze względu na sporą ilość stanów i prostotę automatu zostanie przedstawiony jedynie uproszczony graf przejść (rysunek \ref{lcd_fsm}).


Opis wyprowadzeń:

Wejścia:
\begin{itemize}
\item CLK - wejście zegarowe;
\item SIGN - bit określający znak wyniku;
\item HALF - bit określający wartość po przecinku (0 lub 5);
\item BUSY\_LCD - bit zajętości LCDWrite;
\item S(3:0) - 4 bity cyfry setek;
\item D(3:0) - 4 bity cyfry dziesiątek;
\item J(3:0) - 4 bity cyfry jedności;
\end{itemize}

Wyjścia:
\begin{itemize}
\item WE - jednotaktowy impuls inicjalizujący; 
\item DnI - bit sterujący;
\item Byte(7:0) - wektor wyjściowy znaku do wyświetlenia;
\end{itemize}

\begin{figure}[H]
\begin{center}
\includegraphics[height=15cm]{graphics/lcd_fsm.png}
\end{center}
\caption{Maszyna stanów LCD}
\label{lcd_fsm}
\end{figure}

\subsection{Pozostałe moduły}

Moduły LCDWrite oraz RotaryEnc pochodzą z zasobów dr. Jarosława Sugiera, ich opis znajduje się w \cite{bib_lcdwrite} oraz \cite{bib_rotaryenc}.

\section{Obsługa}

Gotowy układ przedstawia rysunek \ref{photo}. Termometr należy podłączyć do złącza J4. Wynik pomiaru wyświetlony jest na wyświetlaczu LCD. Sterowanie częstotliwością pracy odbywa się przy użyciu enkodera znajdującego się po lewej stronie ekranu. Aktualnie wybraną częstotliwość pokazują diody LED po jego prawej stronie (zapalona skrajna prawa oznacza pomiar ciągły, 4 od prawej co 4 s).

\begin{figure}[H]
\begin{center}
\includegraphics[width=16cm]{graphics/photo.png}
\end{center}
\caption{Zrealizowany układ}
\label{photo}
\end{figure} 

\section{Podsumowanie}
\subsection{Ocena projektu}

W ramach projektu zostały zrealizowane wszystkie założenia ustalone przed przystąpieniem do pracy. Dodatkowo, już po realizacji zamierzeń, dodana została obsługa zmiany częstotliwości pomiaru. Wszystkie moduły napisane w ramach projektu były na bieżąco testowane przy użyciu symulatora ISim, co zapewnia poprawność jego pracy. Ostatecznie, układ może działać z częstotliwością około 57 MHz. Dość niska maksymalna częstotliwość wynika prawdopodobnie z algorytmu double dabble, którego implementacja wymaga złożonej logiki kombinacyjnej. Przed przystąpieniem do ewentualnej rozbudowy projektu należałoby ujednolicić konwencję nazewniczą stanów, sygnałów wewnętrznych oraz wyprowadzeń układów.

\subsection{Możliwości rozbudowy}

Cały projekt zajmuje około 6\% zasobów dostępnych w układzie Spartan 3e. Daje to szerokie możliwości jego rozbudowy. Jedną z nich może być zwiększenie liczby przyłączonych czujników. Dodatkowo można go wzbogacić o obsługę komunikacji np. z komputerem PC z użyciem wybranego interfejsu. Innym rozszerzeniem może być dodanie obłsugi pamięci (np. karta sd), która miałaby za zadanie zapisywać wyniki pomiarów. Tak zapisane wyniki mogłoby być reprezentowane na monitorze VGA w postaci wykresów.

\newpage
\addcontentsline{toc}{section}{Literatura}
\begin{thebibliography}{9}
\bibitem{bib_kom}Komunikacja 1-Wire [dostęp 23.05.2016] \\
\emph  	https://www.maximintegrated.com/en/app-notes/index.mvp/id/126

\bibitem{bib_lcdwrite}Opis modułu LCDWrite [dostęp 23.05.2016] \\
\emph http://www.zsk.ict.pwr.wroc.pl/zsk\_ftp/fpga/\#\_Toc286057186

\bibitem{bib_rotaryenc}Opis modułu RotaryEnc [dostęp 23.05.2016] \\
\emph http://www.zsk.ict.pwr.wroc.pl/zsk\_ftp/fpga/\#\_Toc286057181

\bibitem{bib_spartan}Przewodnik układu FPGA Spartan 3e starter kit [dostęp 23.05.2016] \\
\emph http://www.xilinx.com/support/documentation/boards\_and\_kits/ug230.pdf

\bibitem{bib_ds18s20}Dokumentacja układu DS18S20 [dostęp 23.05.2016] \\
\emph https://datasheets.maximintegrated.com/en/ds/DS18S20.pdf

\bibitem{bib_numeric}Dokumentacja biblioteki NUMERIC\_STD [dostęp 23.05.2016] \\
\emph http://www.csee.umbc.edu/portal/help/VHDL/packages/numeric\_std.vhd

\bibitem{bib_dabble}Opis algorytmu Double dabble [dostęp 23.05.2016] \\
\emph https://en.wikipedia.org/wiki/Double\_dabble

\end{thebibliography}

\end{document}